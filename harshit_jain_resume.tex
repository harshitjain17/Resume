\documentclass{resume} % Use the custom resume.cls style

\usepackage{enumitem}
\usepackage[left=0.35in,top=0.37in,right=0.35in,bottom=0.37in]{geometry} % Document margins
\newcommand{\tab}[1]{\hspace{.2667\textwidth}\rlap{#1}} 
\newcommand{\itab}[1]{\hspace{0em}\rlap{#1}}

\input{glyphtounicode}
\pdfgentounicode=1

\name{Harshit Jain} % Your name

\address{\href{tel:15822039755}{+1 (582) 203-9755} \\
\href{mailto:harshitj.cs@gmail.com}{harshitj.cs@gmail.com} \\
\href{https://www.linkedin.com/in/harshitjain17}{linkedin.com/in/harshitjain17} \\
\href{https://www.github.com/harshitjain17}{github.com/harshitjain17}}

\begin{document}

%----------------------------------------------------------------------------------------
%	EDUCATION SECTION
%----------------------------------------------------------------------------------------

\begin{rSection}{EDUCATION}

{\bf The Pennsylvania State University, University Park PA} \hfill {Expected Graduation: Dec $2024$}\\
{Bachelor of Science in Computer Science}\\
{\bf GPA:} $3.7/4.0$ $\vert$ Dean's List (5/5) $\vert$ Technical Coach @ \href{https://www.codepath.org/en-us/volunteers/technical-interview-coaching}{CodePath} $\vert$ AlgoPSU Captain @ \href{https://acm.psu.edu/algopsu/}{ACM} $\vert$ Resident Assistant\\
{\bf Relevant Coursework:} Data Structures and Algorithms, Math of Machine Learning, Operating Systems Design, Systems Programming, \href{https://www.coursera.org/account/accomplishments/verify/JYX5UB5YP4YD}{Supervised Machine Learning}, \href{https://www.coursera.org/account/accomplishments/records/PBF4QN2KQL4Z}{Advanced Learning Algorithms} (Deep Learning), \href{https://www.coursera.org/account/accomplishments/verify/3W6HA4ZC7UTY}{Generative AI with LLMs}, Theory of Computation, CodePath: (Intermediate+Advanced) Software Engineering, Database Management Systems
\end{rSection}


%----------------------------------------------------------------------------------------
% TECHINICAL STRENGTHS	
%----------------------------------------------------------------------------------------
\begin{rSection}{TECHNICAL SKILLS}

{\bf Programming Languages:} Python, C/C++, JavaScript, Java, HTML/CSS, MATLAB, Verilog, Assembly ($64/32$-bit x$86$)\\
{\bf Frameworks \& Tools:} \href{https://www.coursera.org/account/accomplishments/verify/7L8L8YQTMCPZ}{AWS}, TensorFlow/Keras, Scikit-Learn, Numpy, Node.js, React.js, Next.js, RESTful APIs, \LaTeX, Git\\
{\bf Softwares:} LLMs, MS SQL Server/MySQL/PostgreSQL, Linux/UNIX, SonarQube, Postman, Bitbucket, JIRA

\end{rSection}


% ----------------------------------------------------------------------------------------
% WORK EXPERIENCE SECTION	
% ----------------------------------------------------------------------------------------
\begin{rSection}{WORK EXPERIENCE}

{\bf Software Engineer Intern} \hfill May $2024$ - Present\\
Hughes Network Systems, LLC \hfill Germantown, MD
\begin{itemize}[itemsep = -4pt]
    \item Working within their Aviation business (Aero program) 
\end{itemize}

{\bf Machine Learning Engineer Intern} \hfill Jan $2024$ - May $2024$\\
Materials Research Institute (2DCC-MIP Team), Penn State University \hfill University Park, PA
\begin{itemize}[itemsep = -4pt]
    \item Leveraged 3 SOTA LLMs (GPT-4, Jurassic2, and Llama2) for chatbot-integrated answer retrieval using the RAG Model
    \item Architected a {\bf Next.js} app with end-to-end AWS integration, optimizing {\bf RESTful APIs} for scalability and deployment
    \item Automated Python-based {\bf AWS Lambdas} for video processing and transcription, handling {\bf 50\% user upload surge}
    \item Implemented AWS Rekognition for face recognition, enabling automatic tagging and improving content organization accuracy
    \item Engineered optimal {\bf AWS EC2} instance selection, reducing launch times by {\bf 25\%} through precise instance matching
    \item \underline{Utilized:} Python, AWS, Next.js, LLMs, Deep Learning Models 
\end{itemize}

{\bf Software Engineer Co-op} \hfill May $2023$ - Dec $2023$\\
VIAVI Solutions Inc. \hfill Germantown, MD
\begin{itemize}[itemsep = -4pt]
    \item Collaborated with the $6$-person R\&D team to architect and implement a {\bf Python-based} automated test suite on Linux systems for PNT instruments, ensuring exhaustive test coverage and strict compliance with the SCPI protocol
    \item Debugged PNT unit source code in {\bf C/C++}, leading to a {\bf 55\% reduction} in bugs and a {\bf 30\% increase} in code coverage
    \item Executed 35+ long-term tests on core devices using SCPI commands, ensuring adherence to release-level quality standards
    \item \underline{Utilized:} Python, C/C++, SCPI Protocol, Bitbucket, Confluence, SonarQube, Git, Agile, JIRA
\end{itemize}

{\bf Software Engineer Intern - Research Associate} \hfill May $2022$ - May $2023$\\
Materials Research Institute (2DCC-MIP Team), Penn State University \hfill University Park, PA
\begin{itemize}[itemsep = -4pt]
    \item Engineered a {\bf React.js} app, designing 50+ user-facing features across {\bf 20+ REACT} components (utilized Jest testing)
    \item Integrated an {\bf MS SQL Server} relational database to manage data from {\bf 500+ instruments} across 18+ tables
    \item Automated {\bf Python scripts} for data retrieval, manipulation, and integration with robust error handling and efficient filtering
    \item \underline{Research:} Developed a {\bf Python library} for the {\href{https://github.com/harshitjain17/Raman-Peak-Fitting-Model/}{Raman Fitting model}}, enabling deconvolution of Raman spectra. The library facilitated interactive preprocessing, effective fitting, and data export, {\bf reducing analysis time by 40\%} (tested).
    \item \underline{Utilized:} Python, JavaScript, React.js, Node.js, MS SQL Server, RESTful APIs, Git, HTML/CSS, JIRA
\end{itemize}

\end{rSection} 


% ----------------------------------------------------------------------------------------
% 	PROJECT EXPERIENCE SECTION
% ----------------------------------------------------------------------------------------

\begin{rSection}{PROJECTS}

{\bf Dynamic Memory Allocator }{[C/C++] } \hfill Jan $2024$ - Feb $2024$
\begin{itemize}[itemsep = -4pt]
    \item Designed {\bf custom malloc, free, realloc}; segregated free lists and footer optimization to improve memory management
    \item Achieved a {\bf utilization score of 69\%} and benchmark {\bf throughput at 100\%} across diverse computing environments
\end{itemize}

{\bf HiLite: AI AutoHighlighter }{[Python, Flask, React.js] } \hfill Mar $2023$ - May $2023$
\begin{itemize}[itemsep = -4pt]
    \item Designed an {\bf AI system} that summarizes text using Long Short-Term Memory (LSTM) networks
    \item Trained the model on the training set, using the validation set to monitor its performance and prevent overfitting
\end{itemize}

{\bf mdadm Linear Device }{[C/C++, Linux] } \hfill Feb $2023$ - May $2023$
\begin{itemize}[itemsep = -4pt]
    \item Configured {\bf 16 disks of size 64 KB as a 1MB} linear device, providing users with a unified address space for data access
    \item Implemented mount/unmount operations to the linear device, mitigating potential data loss and system crashes
    \item Designed the read/write functions to set up in the linear device, providing users with comprehensive data access capabilities
    \item Engineered data caching solution to enhance system latency {\bf reduced I/O wait time by 60\%}
\end{itemize}

% {\bf Library Management System }{[Python] }{(\href{https://github.com/harshitjain17/Library-Management-System}{Try it here})} \hfill Nov $2021$ - Dec $2021$
% \begin{itemize}[itemsep = -4pt]
%     \item Built a {\bf Python}-based library management system optimizing advanced tasks in a library setting; passed tests on {\bf 500+ logs}
%     \item Deployed {\bf 10+ advanced features} and functions, including real-time student eligibility checks for book borrowing according to their historic data, calculated pending fines at the end of the log and on a specific day within the log, and more
% \end{itemize}

\end{rSection}
\end{document}