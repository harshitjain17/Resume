\documentclass{resume} % Use the custom resume.cls style

\usepackage{enumitem}
\usepackage[left=0.35in,top=0.37in,right=0.35in,bottom=0.37in]{geometry} % Document margins
\newcommand{\tab}[1]{\hspace{.2667\textwidth}\rlap{#1}} 
\newcommand{\itab}[1]{\hspace{0em}\rlap{#1}}

\input{glyphtounicode}
\pdfgentounicode=1

\name{Harshit Jain} % Your name

\address{\href{tel:15822039755}{+1 (582) 203-9755} \\
\href{mailto:harshitj.cs@gmail.com}{harshitj.cs@gmail.com} \\
\href{https://www.linkedin.com/in/harshitjain17}{linkedin.com/in/harshitjain17} \\
\href{https://www.github.com/harshitjain17}{github.com/harshitjain17}}

\begin{document}

%----------------------------------------------------------------------------------------
%	EDUCATION SECTION
%----------------------------------------------------------------------------------------

\begin{rSection}{Education}

{\bf The Pennsylvania State University, University Park PA} \hfill {Expected Graduation: Dec $2024$}\\
{Bachelor of Science in Computer Science}\\
{\bf GPA:} $3.7/4.0$ $\vert$ Dean's List (5/5) $\vert$ Recipient of \href{https://awardsrecognition.psu.edu/student/undergraduate-scholastic-awards/}{The President Walker Award} $\vert$ AlgoPSU Captain @ \href{https://acm.psu.edu/algopsu/}{ACM} $\vert$ Resident Assistant\\
{\bf Relevant Coursework:} Data Structures \& Algorithms, Artificial Intelligence, \href{https://www.coursera.org/account/accomplishments/verify/JYX5UB5YP4YD}{Supervised Machine Learning}, \href{https://www.coursera.org/account/accomplishments/records/PBF4QN2KQL4Z}{Advanced Learning Algorithms (Deep Learning)}, Systems Programming, Operating Systems Design, Theory of Computation, Programming and Computation II: Data Structures, CodePath: (Intermediate + Advanced) Software Engineering, Database Management Systems
\end{rSection}


%----------------------------------------------------------------------------------------
% TECHINICAL STRENGTHS	
%----------------------------------------------------------------------------------------
\begin{rSection}{TECHNICAL SKILLS}

{\bf Programming Languages:} Python, C/C++, JavaScript, Java, HTML/CSS, MATLAB, Verilog, Assembly ($64/32$-bit x$86$)\\
{\bf Frameworks \& Tools:} AWS, TensorFlow/Keras, Scikit-Learn, Numpy, Node.js, React.js, Next.js, RESTful APIs, \LaTeX, Git\\
{\bf Softwares:} MS SQL Server, MySQL Database System, Linux/UNIX, SonarQube, Postman, Bitbucket, JIRA

\end{rSection}


% ----------------------------------------------------------------------------------------
% WORK EXPERIENCE SECTION	
% ----------------------------------------------------------------------------------------
\begin{rSection}{WORK EXPERIENCE}

{\bf Machine Learning Engineer Intern} \hfill Jan $2024$ - Present\\
2D Crystal Consortium - Materials Innovation Platform (2DCC-MIP, MRI), Penn State University \hfill University Park, PA
\begin{itemize}[itemsep = -4pt]
    \item Working on the ML-centric development of "MaterialsTube", a video aggregator platform for data-driven material discovery. Leveraging AWS backend \& specializing in integrating ML models for enhanced video metadata and content analysis
    \item \underline{Utilized:} Python, AWS, Next.js, TensorFlow, Scikit-Learn, Deep Learning Models 
\end{itemize}

{\bf Software Engineer Co-op} \hfill May $2023$ - Dec $2023$\\
VIAVI Solutions Inc. \hfill Germantown, MD
\begin{itemize}[itemsep = -4pt]
    \item Collaborated with the 6-person R\&D team to design and implement a Python-based automated test suite on Linux systems for the EGR $2.0$ (instrument), ensuring comprehensive test coverage and compliance with the SCPI protocol
    \item Debugged PNT-$62$xx unit's source code in C/C++, resulting in a 55\% reduction in bugs and a 30\% increase in code coverage
    \item Performed $35+$ rigorous short-term and long-term tests on core devices using SCPI commands to uphold release-level quality
    \item \underline{Utilized:} Python, C/C++, SCPI Protocol, Bitbucket, Confluence, SonarQube, Git, Agile, JIRA
\end{itemize}

{\bf Software Engineer Intern - Research Associate} \hfill May $2022$ - May $2023$\\
2D Crystal Consortium - Materials Innovation Platform (2DCC-MIP, MRI), Penn State University \hfill University Park, PA
\begin{itemize}[itemsep = -4pt]
    \item Implemented front-end architecture using React.js to design $50+$ latest user-facing features in $20+$ REACT components with $100\%$ accuracy (tested using JEST), built reusable components, and front-end libraries for continuous development
    \item Integrated MS SQL Server relational database which currently deals with $500+$ instruments' data in $18+$ tables
    \item Developed and tested Python scripts automating the process to retrieve data from various sources, manipulate and analyze the data, filter out irrelevant data, look up similar data in the server, import it into the server, and handle errors gracefully
    \item Developed and completely automated a Python library for the Raman Fitting model, to perform deconvolution on Raman spectra, and enable interactive preprocessing, effective fitting, and export of data files, reducing analysis time by $40\%$ (tested)
    \item \underline{Utilized:} Python, JavaScript, React.js, Node.js, MS SQL Server, RESTful APIs, Git, HTML/CSS, JIRA
\end{itemize}

\end{rSection} 


% ----------------------------------------------------------------------------------------
% 	PROJECT EXPERIENCE SECTION
% ----------------------------------------------------------------------------------------

\begin{rSection}{PROJECTS}

{\bf HiLite: AI AutoHighlighter }{(\href{https://github.com/harshitjain17/HiLite-AIAutoHighlighter}{Try it here})} \hfill Mar $2023$ - May $2023$
\begin{itemize}[itemsep = -4pt]
    \item Designed an AI system that automatically identifies and summarizes text using Long Short-Term Memory (LSTM) networks
    \item Created LSTM-based Encoder and Decoder to create a robust text summarization solution
    \item Trained the model on the training set, using the validation set to monitor its performance and prevent overfitting
    \item Utilized Python, Flask, and React.js for the implementation, ensuring a seamless and user-friendly interface
\end{itemize}

{\bf mdadm Linear Device }{(\href{https://github.com/harshitjain17/mdadm-Linear-Device}{Try it here})} \hfill Feb $2023$ - May $2023$
\begin{itemize}[itemsep = -4pt]
    \item Developed the mdadm tool in $C$ for managing multiple disks in Linux systems
    \item Configured $16$ disks of size $64$ KB as a $1$ MB linear device, providing users with a unified address space for data access
    \item Implemented mount/unmount operations to the linear device, mitigating potential data loss and system crashes
    \item Designed the read/write functions to set up in the linear device, providing users with comprehensive data access capabilities
    \item Engineered data caching solution to enhance system latency reduced I/O wait time by $60\%$
\end{itemize}

{\bf Library Management System }{(\href{https://github.com/harshitjain17/Library-Management-System}{Try it here})} \hfill Nov $2021$ - Dec $2021$
\begin{itemize}[itemsep = -4pt]
    \item Built a Python-based library management system streamlining advanced tasks in a library setting; passed tests on $500+$ logs
    \item Deployed $10+$ advanced features and functions for the system, including real-time student eligibility checks for book borrowing according to their historic data, calculated pending fines at the end of the log and on a specific day within the log, and more
\end{itemize}

\end{rSection} 


% ----------------------------------------------------------------------------------------
% 	LEADERSHIP EXPERIENCE SECTION
% ----------------------------------------------------------------------------------------
% \begin{rSection}{LEADERSHIP} 
% \begin{itemize}
%     \itemsep -3pt {}
%     \item Webmaster (Web Manager) in \href{https://ndl.psu.edu/}{Nittany Data Labs (NDL)} Club @ Penn State \hfill Apr $2022$ - Present
%     \item Resident Assistant (RA) in Pollock Halls Complex @ Penn State \hfill Jun $2022$ - Present
% \end{itemize}
% \end{rSection}

% Engineering data caching solution to optimize read/write performance of the linear device; reduced I/O wait time by 80% and increased throughput by 100%
 

\end{document}