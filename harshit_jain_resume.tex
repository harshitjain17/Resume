\documentclass{resume} % Use the custom resume.cls style

\usepackage[left=0.4 in,top=0.4in,right=0.4 in,bottom=0.4in]{geometry} % Document margins
\newcommand{\tab}[1]{\hspace{.2667\textwidth}\rlap{#1}} 
\newcommand{\itab}[1]{\hspace{0em}\rlap{#1}}

\name{Harshit Jain} % Your name

\address{\href{tel:15822039755}{+1 (582) 203-9755} \\
\href{mailto:harshitj.cs@gmail.com}{harshitj.cs@gmail.com} \\
\href{https://www.linkedin.com/in/harshitjain17}{linkedin.com/in/harshitjain17} \\
\href{https://www.github.com/harshitjain17}{github.com/harshitjain17}}

\begin{document}

%----------------------------------------------------------------------------------------
%	EDUCATION SECTION
%----------------------------------------------------------------------------------------

\begin{rSection}{Education}

{\bf The Pennsylvania State University, University Park PA} \hfill {Aug $2021$ - May $2024$ (Expected)}\\
{Bachelor of Science in Computer Science $\vert$ Minor in Mathematics}\\
{\bf GPA:} $3.75/4.0$ $\vert$ Dean's List (All Semesters) $\vert$ Recipient of \href{https://awardsrecognition.psu.edu/student/undergraduate-scholastic-awards/}{The President Walker Award} $\vert$ Webmaster @ \href{https://ndl.psu.edu/}{NDL} $\vert$ Resident Assistant\\
{\bf Relevant Coursework:} Data Structures \& Algorithms, Systems Programming, Artificial Intelligence, Programming Language Concepts, Programming and Computation II: Data Structures (in Python), OOP with Web-based Applications (in Java), CodePath: Intermediate Software Engineering, Linear Programming, Discrete Mathematics
\end{rSection}


%----------------------------------------------------------------------------------------
% TECHINICAL STRENGTHS	
%----------------------------------------------------------------------------------------
\begin{rSection}{SKILLS}

{\bf Programming Languages:} Python, JavaScript, C/C++, Java, HTML/CSS, MATLAB, Verilog, Assembly ($64/32$-bit x$86$)\\
{\bf Softwares:} Visual Studio, Microsoft SQL Server, MySQL Database System, Linux/UNIX, Postman, Agile, JIRA\\
{\bf Frameworks \& Tools:} GoLang, Node.js, React.js, RESTful APIs, \LaTeX, Git

\end{rSection}


%----------------------------------------------------------------------------------------
% WORK EXPERIENCE SECTION	
%----------------------------------------------------------------------------------------
\begin{rSection}{WORK EXPERIENCE}

\textbf{Software Engineering Intern, Research Associate} \hfill May $2022$ - Present\\
Materials Research Institute (MRI), The Pennsylvania State University \hfill \textit{University Park, PA}
 \begin{itemize}
    \itemsep -3pt {}
     \item Contributed to the development of DMR-FIRST Instrument Tool (NSF-funded project), a full-fledged search engine for researchers across the U.S. to locate signature research instruments at their nearest location
     \item Implemented front-end architecture using React.js to design $50+$ latest user-facing features with $100\%$ accuracy (tested using JEST) and built reusable components and front-end libraries for continuous development
     \item Tested and optimized $20+$ REACT components for the best performance across every device and browser
     \item Integrated MS SQL Server relational database which currently deals with $500+$ instruments and $18+$ tables
     \item Saved weeks of development efforts by integrating leverageable modular code by using popular REACT libraries such as Material UI, improving the efficiency by approximately $40\%$
     \item Developed and tested Python scripts automating the process to retrieve data from various sources, manipulate and analyze the data, filter out irrelevant data, look up similar data in the server, import it into the server, and handle errors gracefully
     \item Collaborated and networked effectively with team members at each step of the design process to accomplish the sprint goals 
     \item \underline{Utilized:} Python, JavaScript, React.js, Node.js, MS SQL Server, RESTful APIs, Git, HTML/CSS, JIRA, Office 365
    \end{itemize}

\end{rSection} 


%----------------------------------------------------------------------------------------
%	PROJECT EXPERIENCE SECTION
%----------------------------------------------------------------------------------------

\begin{rSection}{PROJECT EXPERIENCE}

\textbf{ \texttt{mdadm} Linear Device } \hfill Feb $2023$ - Present
\begin{itemize}
    \itemsep -3pt {}
    \item Developed \verb|mdadm| tool in $C$ for managing multiple disks in Linux systems
    \item Configured $16$ disks of size $64$ KB as a $1$ MB linear device, providing users with a unified address space for data access
    \item Implemented mount/unmount operations to the linear device, preventing potential data loss and system crashes
    \item Designed the read/write functions to set up in the linear device, providing users with comprehensive data access capabilities
    \item Rolled out robust error-checking mechanisms including checks for out-of-bounds access and large read-and-write operations

\end{itemize}

\textbf{ Course Scheduler }{\href{https://github.com/harshitjain17/OOP-with-Web-based-Applications-in-Java-CMPSC-221-assignments/tree/main/CourseSchedulerHarshitJainhmj5262}{(Try it here)}} \hfill Oct $2022$ - Dec $2022$
\begin{itemize}
    \itemsep -3pt {}
    \item Developed the application using Java and SQL to allow students and educators to manage their course schedules, including setting up semesters, adding courses, and new students
    \item Designed the Derby Database to store and organize the data, and wrote several SQL queries to facilitate tasks such as adding and dropping courses, and managing the waitlist process
\end{itemize}


\textbf{Library Management System }{\href{https://github.com/harshitjain17/Library-Management-System}{(Try it here)}} \hfill Nov $2021$ - Dec $2021$
\begin{itemize}
    \itemsep -3pt {}
    \item Built a Python-based professional library management system that streamlines basic and advanced tasks in a library setting; the system passed tests on $500+$ library logs
    \item Developed multiple features and functions for the system, including:
    \begin{itemize}
        \itemsep -3pt {}
        \item[$*$] Checks the eligibility of students to borrow books on a particular day for a certain number of days
        \item[$*$] Finds the most borrowed/popular books in the library
        \item[$*$] Finds the book that has the highest borrow ratio
        \item[$*$] Produces a sorted list of the most borrowed books (books with the highest usage ratio)
        \item[$*$] Calculates the pending fines at the end of the log and on a specific day in the log
    \end{itemize}
\end{itemize}

\end{rSection} 


%----------------------------------------------------------------------------------------
%	LEADERSHIP EXPERIENCE SECTION
%----------------------------------------------------------------------------------------
% \begin{rSection}{LEADERSHIP} 
% \begin{itemize}
%     \itemsep -3pt {}
%     \item Webmaster (Web Manager) in \href{https://ndl.psu.edu/}{Nittany Data Labs (NDL)} Club @ Penn State \hfill Apr $2022$ - Present
%     \item Resident Assistant (RA) in Pollock Halls Complex @ Penn State \hfill Jun $2022$ - Present
% \end{itemize}

% \end{rSection}


\end{document}