\documentclass{resume} % Use the custom resume.cls style

\usepackage{enumitem}
\usepackage[left=0.35in,top=0.37in,right=0.35in,bottom=0.37in]{geometry} % Document margins
\newcommand{\tab}[1]{\hspace{.2667\textwidth}\rlap{#1}} 
\newcommand{\itab}[1]{\hspace{0em}\rlap{#1}}

\input{glyphtounicode}
\pdfgentounicode=1

\name{Harshit Jain} % Your name

\address{\href{tel:15822039755}{+1 (582) 203-9755} \\
\href{mailto:harshitj.cs@gmail.com}{harshitj.cs@gmail.com} \\
\href{https://www.linkedin.com/in/harshitjain17}{linkedin.com/in/harshitjain17} \\
\href{https://www.github.com/harshitjain17}{github.com/harshitjain17}}

\begin{document}

%----------------------------------------------------------------------------------------
%	EDUCATION SECTION
%----------------------------------------------------------------------------------------

\begin{rSection}{EDUCATION}

{\bf The Pennsylvania State University, University Park PA} \hfill {Expected Graduation: Dec $2024$}\\
{Bachelor of Science in Computer Science}\\
{\bf GPA:} $3.7/4.0$ $\vert$ Dean's List (5/5) $\vert$ Technical Coach @ \href{https://www.codepath.org/en-us/volunteers/technical-interview-coaching}{CodePath} $\vert$ AlgoPSU Captain @ \href{https://acm.psu.edu/algopsu/}{ACM} $\vert$ \href{https://www.linkedin.com/in/harshitjain17/details/recommendations/}{Recommendations} $\vert$ \href{https://www.linkedin.com/in/harshitjain17/details/projects/}{Projects} $\vert$ RA \\
{\bf Relevant Coursework:} Data Structures and Algorithms, Math of Machine Learning, Operating Systems Design, Systems Programming, \href{https://www.coursera.org/account/accomplishments/verify/JYX5UB5YP4YD}{Supervised Machine Learning}, \href{https://www.coursera.org/account/accomplishments/records/PBF4QN2KQL4Z}{Advanced Learning Algorithms} (Deep Learning), \href{https://www.coursera.org/account/accomplishments/verify/3W6HA4ZC7UTY}{Generative AI with LLMs}, Theory of Computation, CodePath: (Intermediate+Advanced) Software Engineering, Database Management Systems, Financial Engineering

\end{rSection}


%----------------------------------------------------------------------------------------
% TECHINICAL STRENGTHS	
%----------------------------------------------------------------------------------------
\begin{rSection}{TECHNICAL SKILLS}

{\bf Programming Languages:} Python, C/C++, JavaScript, Java, HTML/CSS, MATLAB, Verilog, Assembly ($64/32$-bit x$86$)\\
{\bf Frameworks \& Tools:} TensorFlow, Scikit-Learn, NumPy, Node.js, React.js, Next.js, GraphQL, RESTful APIs, LaTeX, Git\\
{\bf Softwares:} \href{https://www.coursera.org/account/accomplishments/verify/7L8L8YQTMCPZ}{AWS}, GCP, LLMs, MS SQL Server/MySQL/PostgreSQL, Linux/UNIX, SonarQube, Postman, Bitbucket, JIRA

\end{rSection}


% ----------------------------------------------------------------------------------------
% WORK EXPERIENCE SECTION	
% ----------------------------------------------------------------------------------------
\begin{rSection}{WORK EXPERIENCE}

{\bf Software Engineer Intern} \hfill Aug $2024$ - Present\\
Materials Research Institute (2DCC-MIP Team), Penn State University \hfill University Park, PA
    
{\bf Software Engineer Intern} \hfill May $2024$ - Aug $2024$\\
Hughes Network Systems, LLC (Aeronautical Team) \hfill Germantown, MD
\begin{itemize}[itemsep = -4pt]
    \item Automated AWS-to-BigQuery pipeline with strict row-level validation \& GCS audit trails, cutting manual processes by 95\%
    \item Developed Python frameworks and BigQuery SQL views for monthly KPI evaluation of aircraft and fleet performance, processing data from 61k+ Delta Airlines flights with O(1) time complexity and sub-10-second execution
    \item Designed an automated script for flight performance billing and dynamic performance adjustments for accurate invoicing
    \item \underline{Technologies Used:} Python, Google Cloud Platform (GCP: Cloud Functions, BigQuery, Pub/Sub), SQL, Agile/Scrum
\end{itemize}

{\bf Machine Learning Engineer Intern} \hfill Jan $2024$ - May $2024$\\
Materials Research Institute (2DCC-MIP Team), Penn State University \hfill University Park, PA
\begin{itemize}[itemsep = -4pt]
    \item Integrated GPT-4, Jurassic2, Llama2 LLMs into a chatbot framework (LangChain) for answer retrieval via an RAG model 
    \item Built a serverless Next.js app with AWS integration, optimizing RESTful APIs for high scalability and CI/CD workflows
    \item Developed a fully automated AWS Lambda pipeline for video processing, including transcription and AI-generated metadata
    \item Configured AWS Rekognition to boost speaker identification, improve transcription, and automate chapter generation
    \item \underline{Technologies Used:} Python, LangChain, Amazon Web Services (AWS), Next.js, LLMs, Deep Learning Models 
\end{itemize}

{\bf Software Engineer Co-op} \hfill May $2023$ - Dec $2023$\\
VIAVI Solutions Inc. \hfill Germantown, MD
\begin{itemize}[itemsep = -4pt]
    \item Engineered a Python-based automated test suite for PNT instruments on Linux, in collaboration with a 6-member R\&D team, to validate system performance and ensure strict adherence to the SCPI protocol
    \item Troubleshot PNT unit bugs, reducing by 55\% and increasing code coverage by 30\% (SonarQube) through C/C++ debugging
    \item Executed 35+ SCPI-driven long-term tests on core devices to ensure compliance with release-level quality standards
    \item \underline{Technologies Used:} Python, C/C++, Linux, SCPI Protocol, Bitbucket, Confluence, SonarQube, Git, JIRA, Agile/Scrum
\end{itemize}

{\bf Software Engineer Intern} \hfill May $2022$ - May $2023$\\
Materials Research Institute (2DCC-MIP Team), Penn State University \hfill University Park, PA
\begin{itemize}[itemsep = -4pt]
    \item Engineered a scalable React.js app with 50+ user-facing features across 20+ components, and employed Jest for unit testing
    \item Integrated MS SQL Server to manage data from 500+ instruments, and organized it across 18+ BCNF-normalized tables
    \item Built a Python script for automated data retrieval and integration with a cloud-based data warehouse using ETL processes
    \item \underline{Research:} Re-architected a Python library for the Raman Fitting Model ({\href{https://github.com/harshitjain17/Raman-Peak-Fitting-Model/}{try here}}) to implement precise spectra deconvolution algorithms with automated data preprocessing, fitting, and export functions, reducing analysis time by 40\% (validated)
    \item \underline{Technologies Used:} Python, JavaScript, React.js, Node.js, MS SQL Server, RESTful APIs, Git, HTML/CSS, JIRA
\end{itemize}

\end{rSection} 


% ----------------------------------------------------------------------------------------
% 	PROJECT EXPERIENCE SECTION
% ----------------------------------------------------------------------------------------

\begin{rSection}{PROJECTS}

{\bf Dynamic Memory Allocator }{[C/C++] } \hfill Jan $2024$ - Feb $2024$
\begin{itemize}[itemsep = -4pt]
    \item Designed custom malloc, free, realloc; segregated free lists and footer optimization to improve memory management
    \item Achieved a utilization score of 69\% and benchmark throughput at 100\% across diverse computing environments
\end{itemize}

{\bf mdadm Linear Device }{[C/C++, Linux] } \hfill Feb $2023$ - May $2023$
\begin{itemize}[itemsep = -4pt]
    \item Configured 16 disks of size 64 KB as a 1MB linear device, providing users with a unified address space for data access
    \item Implemented mount/unmount operations to the linear device, mitigating potential data loss and system crashes
    \item Designed the read/write functions and engineered data caching solutions reducing I/O wait time by 60\%
\end{itemize}

\end{rSection}
\end{document}