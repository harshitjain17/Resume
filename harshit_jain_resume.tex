\documentclass{resume} % Use the custom resume.cls style

\usepackage{enumitem}
\usepackage[left=0.35in,top=0.37in,right=0.35in,bottom=0.37in]{geometry} % Document margins
\newcommand{\tab}[1]{\hspace{.2667\textwidth}\rlap{#1}} 
\newcommand{\itab}[1]{\hspace{0em}\rlap{#1}}

\input{glyphtounicode}
\pdfgentounicode=1

\name{Harshit Jain} % Your name

\address{\href{tel:15822039755}{+1 (582) 203-9755} \\
\href{mailto:harshitj.cs@gmail.com}{harshitj.cs@gmail.com} \\
\href{https://www.linkedin.com/in/harshitjain17}{linkedin.com/in/harshitjain17} \\
\href{https://www.github.com/harshitjain17}{github.com/harshitjain17}}

\begin{document}

%----------------------------------------------------------------------------------------
%	EDUCATION SECTION
%----------------------------------------------------------------------------------------

\begin{rSection}{EDUCATION}

{\bf The Pennsylvania State University, University Park PA} \hfill {Expected Graduation: Dec $2024$}\\
{Bachelor of Science in Computer Science}\\
{\bf GPA:} $3.7/4.0$ $\vert$ Teaching Assistant @\href{https://www.codepath.org/en-us/volunteers/technical-interview-coaching}{CodePath} $\vert$ AlgoPSU Captain @\href{https://acm.psu.edu/algopsu/}{ACM} $\vert$ \href{https://www.linkedin.com/in/harshitjain17/details/recommendations/}{Recommendations} $\vert$ \href{https://www.linkedin.com/in/harshitjain17/details/projects/}{Projects} $\vert$ Resident Assistant \\
{\bf Coursework:} Data Structures and Algorithms, Operating Systems, Systems Programming, CodePath: (Intermediate+Advanced) Software Engineering, Computer Vision, Theory of Computation, Database Management Systems, Financial Engineering \\
{\bf Certifications:}  \href{https://www.coursera.org/account/accomplishments/verify/7L8L8YQTMCPZ}{AWS}, \href{https://www.coursera.org/account/accomplishments/verify/JYX5UB5YP4YD}{Machine Learning}, \href{https://www.coursera.org/account/accomplishments/records/PBF4QN2KQL4Z}{Advanced Learning Algorithms} (Deep Learning), \href{https://www.coursera.org/account/accomplishments/verify/3W6HA4ZC7UTY}{Generative AI with LLMs}
\end{rSection}


%----------------------------------------------------------------------------------------
% TECHINICAL STRENGTHS	
%----------------------------------------------------------------------------------------
\begin{rSection}{TECHNICAL SKILLS}

{\bf Programming Languages:} Python, C/C++, JavaScript/TypeScript, SQL, Java, Shell scripting, MATLAB\\
{\bf Frameworks:} TensorFlow, Scikit-Learn, LangChain, Streamlit, Node.js, React.js, Next.js, GraphQL, RESTful APIs, Git\\
{\bf Softwares:} AWS, GCP, Azure, MS SQL Server/PostgreSQL, Linux/UNIX, Docker, SonarQube, Postman, Swagger, JIRA

\end{rSection}


% ----------------------------------------------------------------------------------------
% WORK EXPERIENCE SECTION	
% ----------------------------------------------------------------------------------------
\begin{rSection}{WORK EXPERIENCE}

{\bf Software Engineer Intern} \hfill Aug $2024$ - Present\\
Materials Research Institute (2DCC-MIP Team) \hfill University Park, PA
\begin{itemize}[itemsep = -4pt]
    \item Co-built a scalable platform (Next.js) to deliver low-latency ML inference on AWS for real-time AI-driven video processing
    \item Optimized video retrieval by designing GraphQL APIs to enhance data handling via DynamoDB integration for CRUD Ops
    \item Enabled secure CRUD Ops in a Next.js app by connecting AWS API Gateway to Lambda, authenticated with AWS Cognito
    \item Deployed an agent workflow (Python) with LangChain and utilized Pinecone to convert video metadata to vector embeddings
    \item \underline{Technologies Used:} Python, Amazon Web Services (AWS), LangChain, Pinecone, Next.js, GraphQL, REST APIs
\end{itemize}

{\bf Software Engineer Intern} \hfill May $2024$ - Aug $2024$\\
Hughes Network Systems, LLC (Aeronautical Team) \hfill Washington, DC
\begin{itemize}[itemsep = -4pt]
    \item Automated AWS-to-BigQuery pipeline with strict row-level validation \& GCS audit trails, cutting manual processes by 95\%
    \item Developed Python scripts and optimized BigQuery SQL views to improve flight performance metrics for 61k+ flights' data
    \item Designed an automated billing script on GCP for flight performance to refine invoicing accuracy through dynamic adjustments
    \item \underline{Technologies Used:} Python, Google Cloud Platform (GCP: Cloud Functions, BigQuery, Pub/Sub), SQL, Agile/Scrum
\end{itemize}

{\bf Software Engineer Intern} \hfill Jan $2024$ - May $2024$\\
Materials Research Institute (2DCC-MIP Team) \hfill University Park, PA
\begin{itemize}[itemsep = -4pt]
    \item Integrated GPT-4o and Llama2 LLMs into a LangChain-based chatbot framework for answer retrieval via an RAG model 
    \item Built a serverless Next.js app integrated with AWS and streamlined RESTful APIs for high scalability and CI/CD workflows
    \item Developed a fully automated AWS Lambda pipeline for video processing, including transcription and AI-generated metadata
    \item Configured AWS Rekognition to enhance speaker identification, improve transcription, and automate chapter generation
    \item \underline{Technologies Used:} Python, Amazon Web Services (AWS), LangChain, Next.js, RESTful APIs, LLMs
\end{itemize}

{\bf Software Engineer Co-op} \hfill May $2023$ - Dec $2023$\\
VIAVI Solutions Inc. (PNT Team) \hfill Washington, DC
\begin{itemize}[itemsep = -4pt]
    \item Engineered a Python-based automated test suite for PNT instruments on Linux, in collaboration with a 6-member R\&D team, to validate system performance and ensure strict adherence to the SCPI protocol
    \item Troubleshot PNT unit bugs, reducing by 55\% and increasing code coverage by 30\% (SonarQube) through C/C++ debugging
    \item Executed 35+ SCPI-driven long-term tests on core devices to ensure compliance with release-level quality standards
    \item \underline{Technologies Used:} Python, C/C++, Linux, SCPI Protocol, Bitbucket, Confluence, SonarQube, Git, JIRA, Agile/Scrum
\end{itemize}

{\bf Software Engineer Intern} \hfill May $2022$ - May $2023$\\
Materials Research Institute (2DCC-MIP Team) \hfill University Park, PA
\begin{itemize}[itemsep = -4pt]
    \item Engineered a scalable React.js app with 50+ user-facing features across 20+ components, and employed Jest for unit testing
    \item Integrated MS SQL Server to manage data from 500+ instruments, and organized it across 18+ BCNF-normalized tables
    \item Built a Python-based ETL pipeline to automate data extraction and integrated the data with a cloud-based data warehouse
    \item \underline{Research:} Re-architected a Python library for the Raman Fitting Model ({\href{https://github.com/harshitjain17/Raman-Peak-Fitting-Model/}{try here}}) to implement precise spectra deconvolution algorithms with automated data preprocessing, fitting, and export functions, reducing analysis time by 40\% (validated)
    \item \underline{Technologies Used:} Python, JavaScript, React.js, Node.js, MS SQL Server, RESTful APIs, Git, HTML/CSS, JIRA
\end{itemize}

\end{rSection} 


% ----------------------------------------------------------------------------------------
% 	PROJECT EXPERIENCE SECTION
% ----------------------------------------------------------------------------------------

\begin{rSection}{PROJECTS}

{\bf Dynamic Memory Allocator }{[C/C++] } \hfill Jan $2024$ - Feb $2024$
\begin{itemize}[itemsep = -4pt]
    \item Designed custom malloc, free, realloc; segregated free lists and footer optimization to improve memory management
    \item Achieved a utilization score of 69\% and benchmark throughput at 100\% across diverse computing environments
\end{itemize}

\end{rSection}
\end{document}