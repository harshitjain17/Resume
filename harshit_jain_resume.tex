\documentclass{resume} % Use the custom resume.cls style

\usepackage{enumitem}
\usepackage[left=0.3in,top=0.3in,right=0.3in,bottom=0.3in]{geometry} % Document margins
\newcommand{\tab}[1]{\hspace{.2667\textwidth}\rlap{#1}} 
\newcommand{\itab}[1]{\hspace{0em}\rlap{#1}}

\name{Harshit Jain} % Your name

\address{\href{tel:15822039755}{+1 (582) 203-9755} \\
\href{mailto:harshitj.cs@gmail.com}{harshitj.cs@gmail.com} \\
\href{https://www.linkedin.com/in/harshitjain17}{linkedin.com/in/harshitjain17} \\
\href{https://www.github.com/harshitjain17}{github.com/harshitjain17}}

\begin{document}

%----------------------------------------------------------------------------------------
%	EDUCATION SECTION
%----------------------------------------------------------------------------------------

\begin{rSection}{Education}

{\bf The Pennsylvania State University, University Park PA} \hfill {Expected Graduation: Dec $2024$}\\
{Bachelor of Science in Computer Science $\vert$ Minor in Mathematics}\\
{\bf GPA:} $3.7/4.0$ $\vert$ Dean's List (All Semesters) $\vert$ Recipient of \href{https://awardsrecognition.psu.edu/student/undergraduate-scholastic-awards/}{The President Walker Award} $\vert$ Webmaster @ \href{https://ndl.psu.edu/}{NDL} $\vert$ Resident Assistant\\
{\bf Relevant Coursework:} Data Structures \& Algorithms, Systems Programming, Artificial Intelligence, Programming Language Concepts, Programming and Computation II: Data Structures (in Python), OOP with Web-based Applications (in Java), CodePath: (Intermediate + Advanced) Software Engineering, Linear Programming, Discrete Mathematics
\end{rSection}


%----------------------------------------------------------------------------------------
% TECHINICAL STRENGTHS	
%----------------------------------------------------------------------------------------
\begin{rSection}{TECHNICAL SKILLS}

{\bf Programming Languages:} Python, JavaScript, C/C++, Java, HTML/CSS, MATLAB, Verilog, Assembly ($64/32$-bit x$86$)\\
{\bf Frameworks \& Tools:} Node.js, React.js, React Native, RESTful APIs, \LaTeX, Git, Agile (Scrum) Methodology\\
{\bf Softwares:} Visual Studio, Microsoft SQL Server, MySQL Database System, Linux/UNIX, Postman, Bitbucket, JIRA

\end{rSection}


%----------------------------------------------------------------------------------------
% WORK EXPERIENCE SECTION	
%----------------------------------------------------------------------------------------
\begin{rSection}{WORK EXPERIENCE}

\textbf{Software Engineering Co-op} \hfill May $2023$ - Present\\
VIAVI Solutions Inc. \hfill Germantown, MD
 \begin{itemize}[itemsep = -4pt]
     \item Collaborating with the R\&D team to design and implement a Python-based automated test suite on Linux systems for the EGR $2.0$ (instrument), ensuring comprehensive test coverage and compliance with the SCPI protocol
     \item Debugged PNT-$62$xx unit's source code in C, resulting in a 30\% reduction in bugs and a 20\% increase in code coverage
     \item Upgrading multiple firmware for blocks in $3$x PNT-$6250$ Rubidium units, ensuring compliance with business requirements
     \item Coordinated automation and security practices to support an Agile Development Engineering environment utilizing Scrum
     \item \underline{Utilized:} Python, C/C++, SCPI Protocol, Bitbucket, Confluence, Git, Agile, JIRA, Office 365
    \end{itemize}

\textbf{Software Engineering Intern - Research Associate} \hfill May $2022$ - May $2023$\\
Materials Research Institute (MRI), The Pennsylvania State University \hfill University Park, PA
 \begin{itemize}[itemsep = -4pt]
     \item Implemented front-end architecture using React.js to design $50+$ latest user-facing features in $20+$ REACT components with $100\%$ accuracy (tested using JEST), built reusable components, and front-end libraries for continuous development
     \item Saved weeks of development efforts during a team sprint by integrating leverageable modular code by using the popular REACT libraries such as Material UI, improving the efficiency by approximately $40\%$
     \item Integrated MS SQL Server relational database which currently deals with $500+$ instruments and $18+$ tables
     \item Developed and tested Python scripts automating the process to retrieve data from various sources, manipulate and analyze the data, filter out irrelevant data, look up similar data in the server, import it into the server, and handle errors gracefully
     \item Developed and completely automated a Python library for the Raman Fitting model, to perform deconvolution on Raman spectra, and enable interactive preprocessing, effective fitting, and export of data files, reducing analysis time by $40\%$
     \item \underline{Utilized:} Python, JavaScript, React.js, Node.js, MS SQL Server, RESTful APIs, Git, HTML/CSS, JIRA, Office 365
    \end{itemize}

\end{rSection} 


%----------------------------------------------------------------------------------------
%	PROJECT EXPERIENCE SECTION
%----------------------------------------------------------------------------------------

\begin{rSection}{PROJECTS}

\textbf{ HiLite: AI AutoHighlighter }{(\href{https://github.com/harshitjain17/HiLite-AIAutoHighlighter}{Try it here})} \hfill Mar $2023$ - May $2023$
\begin{itemize}[itemsep = -4pt]
    \item Designed and implemented an AI system that automatically identifies and summarizes text using LSTM networks
    \item Created LSTM-based Encoder and Decoder to create a robust text summarization solution
    \item Trained the model on the training set, using the validation set to monitor its performance and prevent overfitting
    \item Utilized Python, Flask, and React.js for the implementation, ensuring a seamless and user-friendly interface
\end{itemize}

\textbf{ \texttt{mdadm} Linear Device }{(\href{https://github.com/harshitjain17/mdadm-Linear-Device}{Try it here})} \hfill Feb $2023$ - May $2023$
\begin{itemize}[itemsep = -4pt]
    \item Developed \verb|mdadm| tool in $C$ for managing multiple disks in Linux systems
    \item Configured $16$ disks of size $64$ KB as a $1$ MB linear device, providing users with a unified address space for data access
    \item Implemented mount/unmount operations to the linear device, preventing potential data loss and system crashes
    \item Designed the read/write functions to set up in the linear device, providing users with comprehensive data access capabilities
    \item Implemented caching feature which significantly enhances system latency and reduces the load on the JBOD
    \item Added JBOD Networking feature to enable communication with JBOD servers over the network and allow seamless switching to alternative JBOD systems in case of malfunctions
\end{itemize}

\textbf{Library Management System }{(\href{https://github.com/harshitjain17/Library-Management-System}{Try it here})} \hfill Nov $2021$ - Dec $2021$
\begin{itemize}[itemsep = -4pt]
    \item Built a Python-based professional library management system that streamlines basic and advanced tasks in a library setting; the system passed tests on $500+$ library logs
    \item Developed multiple features and functions for the system, including:
    \begin{itemize}[itemsep = -4pt]
        \item[$*$] Checks the eligibility of students to borrow books on a particular day for a certain number of days
        \item[$*$] Finds the book that has the highest borrowing ratio
        \item[$*$] Produces a sorted list of the most borrowed books (books with the highest usage ratio)
        \item[$*$] Calculates the pending fines at the end of the log and on a specific day in the log
    \end{itemize}
\end{itemize}

\end{rSection} 


% ----------------------------------------------------------------------------------------
% 	LEADERSHIP EXPERIENCE SECTION
% ----------------------------------------------------------------------------------------
% \begin{rSection}{LEADERSHIP} 
% \begin{itemize}
%     \itemsep -3pt {}
%     \item Webmaster (Web Manager) in \href{https://ndl.psu.edu/}{Nittany Data Labs (NDL)} Club @ Penn State \hfill Apr $2022$ - Present
%     \item Resident Assistant (RA) in Pollock Halls Complex @ Penn State \hfill Jun $2022$ - Present
% \end{itemize}

% \end{rSection}


\end{document}