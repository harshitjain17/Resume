\documentclass{resume} % Use the custom resume.cls style

\usepackage{enumitem}
\usepackage[left=0.35in,top=0.37in,right=0.35in,bottom=0.37in]{geometry} % Document margins
\newcommand{\tab}[1]{\hspace{.2667\textwidth}\rlap{#1}} 
\newcommand{\itab}[1]{\hspace{0em}\rlap{#1}}

\input{glyphtounicode}
\pdfgentounicode=1

\name{Harshit Jain} % Your name

\address{\href{tel:15822039755}{(582) 203-9755} \\
\href{mailto:harshitj.cs@gmail.com}{harshitj.cs@gmail.com} \\
\href{https://www.linkedin.com/in/harshitjain17}{linkedin.com/in/harshitjain17} \\
\href{https://www.github.com/harshitjain17}{github.com/harshitjain17} \\
\href{https://leetcode.com/u/harshitjain17/}{leetcode.com/harshitjain17}}

\begin{document}

%----------------------------------------------------------------------------------------
%	EDUCATION SECTION
%----------------------------------------------------------------------------------------

\begin{rSection}{EDUCATION}

{\bf The Pennsylvania State University, University Park PA} \hfill {Graduation: Dec $2024$}\\
{Bachelor of Science in Computer Science} $\vert$ GPA: $3.7/4.0$ $\vert$ Teaching Assistant @\href{https://www.codepath.org/en-us/volunteers/technical-interview-coaching}{CodePath} $\vert$ \href{https://www.linkedin.com/in/harshitjain17/details/recommendations/}{Recommendations} $\vert$ \href{https://www.linkedin.com/in/harshitjain17/details/projects/}{Projects}\\
% {\bf GPA:} $3.7/4.0$ $\vert$ Teaching Assistant @\href{https://www.codepath.org/en-us/volunteers/technical-interview-coaching}{CodePath} $\vert$ AlgoPSU Captain @\href{https://acm.psu.edu/algopsu/}{ACM} $\vert$ \href{https://www.linkedin.com/in/harshitjain17/details/recommendations/}{Recommendations} $\vert$ \href{https://www.linkedin.com/in/harshitjain17/details/projects/}{Projects} $\vert$ Resident Assistant \\
{\bf Course:} Data Structures \& Algorithms, Operating Systems, Object-Oriented Programming, Linear Programs, Database Systems \\
{\bf Certifications:} \href{https://learn.microsoft.com/api/credentials/share/en-us/harshitjain17/36D25FDF9BE21293?sharingId=87F1F438F3879B8C}{Microsoft: Azure AI Engineer Associate}, \href{https://www.coursera.org/account/accomplishments/verify/7L8L8YQTMCPZ}{AWS}, \href{https://www.coursera.org/account/accomplishments/verify/JYX5UB5YP4YD}{Machine Learning}, \href{https://www.linkedin.com/in/harshitjain17/details/education/}{CodePath: Advanced Software Engineering}
\end{rSection}


%----------------------------------------------------------------------------------------
% TECHINICAL STRENGTHS	
%----------------------------------------------------------------------------------------
\begin{rSection}{TECHNICAL SKILLS}

{\bf Programming Languages:} Python, Java, C/C++, JavaScript/TypeScript, SQL, Shell scripting\\
{\bf Frameworks:} Streamlit, Node.js, React.js, Next.js, LangChain, TensorFlow, Scikit-Learn, GraphQL, RESTful APIs, Git\\
{\bf Softwares:} AWS, GCP, Azure, MS SQL Server/PostgreSQL, Linux/UNIX, Docker, Postman, CI/CD (Jenkins, Cloud Build)

\end{rSection}


% ----------------------------------------------------------------------------------------
% WORK EXPERIENCE SECTION	
% ----------------------------------------------------------------------------------------
\begin{rSection}{WORK EXPERIENCE}

{\bf Software Engineer} \hfill Jan $2025$ - Present\\
Hughes Network Systems, LLC (Aeronautical Team) \hfill Washington, DC
\begin{itemize}[itemsep = -4pt]
    \item Built a Python-based flight reconciliation system with multi-source matching to automate billing and prevent revenue leakage
\end{itemize}

{\bf Software Engineer Intern} \hfill Aug $2024$ - Dec $2024$\\
Materials Research Institute (2DCC-MIP Team) \hfill University Park, PA
\begin{itemize}[itemsep = -4pt]
    \item Engineered an AI-powered video platform (Next.js, AWS) enabling 10+ U.S. institutes to streamline research video analysis
    \item Reduced video processing latency by 20\% by optimizing GraphQL APIs and integrating DynamoDB for faster video retrieval
    \item Designed a secure authentication flow (AWS Cognito, API Gateway) to enable controlled CRUD access for research datasets
    \item Deployed a Python AI pipeline (LangChain, Pinecone) to enable intelligent search across 1k+ videos using vector embeddings
    \item \underline{Technologies Used:} Python, Amazon Web Services (AWS), LangChain, Pinecone, Next.js, GraphQL, REST APIs
\end{itemize}

{\bf Software Engineer Intern} \hfill May $2024$ - Aug $2024$\\
Hughes Network Systems, LLC (Aeronautical Team) \hfill Washington, DC
\begin{itemize}[itemsep = -4pt]
    \item Automated an AWS-to-BigQuery pipeline (Python) with strict row-level validation and reduced manual intervention by 95\%
    \item Delivered SQL views and a Python-based analytics pipeline to power Delta Airlines’ flight insights for 61k+ flights monthly
    \item Built a Python-based automated invoicing system (GCP) to drive multi-stream revenue capture for flight operations at scale
    \item \underline{Technologies Used:} Python, Google Cloud Platform (GCP: Cloud Functions, BigQuery, Pub/Sub), SQL, Agile/Scrum
\end{itemize}

{\bf Software Engineer Intern} \hfill Jan $2024$ - May $2024$\\
Materials Research Institute (2DCC-MIP Team) \hfill University Park, PA
\begin{itemize}[itemsep = -4pt]
    \item Delivered an AI-driven video pipeline using AWS Lambda to automate transcription and tagging for 1,000+ hours of content
    \item Integrated LLMs (GPT-4, Claude 3) into a LangChain chatbot to enable RAG-based retrieval for 800+ queries monthly
    \item Built a video classification system with AWS Rekognition to automate speaker identification and cut manual tagging by 75\%
    \item Developed an automated CI/CD pipeline for serverless research apps, ensuring rapid deployments and 99.9\% uptime reliability
    \item \underline{Technologies Used:} Python, Amazon Web Services (AWS), LangChain, Next.js, RESTful APIs, LLMs
\end{itemize}

{\bf Software Engineer Co-op} \hfill May $2023$ - Dec $2023$\\
VIAVI Solutions Inc. (PNT Team) \hfill Washington, DC
\begin{itemize}[itemsep = -4pt]
    \item Built a distributed Python/C++ test framework for PNT products to cut testing time by 70\% and speed up market delivery
    \item Redesigned a multi-tiered system to microservices architecture to resolve SCPI bottlenecks and boost response time by 40\%
    \item Executed 35+ SCPI compliance tests to reduce product validation time from 2 weeks to 3 days for critical product releases
    \item \underline{Technologies Used:} Python, C/C++, Linux, SCPI Protocol, Bitbucket, Confluence, SonarQube, Git, JIRA, Agile/Scrum
\end{itemize}

{\bf Software Engineer Intern} \hfill May $2022$ - May $2023$\\
Materials Research Institute (2DCC-MIP Team) \hfill University Park, PA
\begin{itemize}[itemsep = -4pt]
    \item Built scalable React.js application with 50+ features and Jest test coverage to accelerate data analysis for 200+ scientists
    \item Integrated MS SQL Server to manage data from 900+ instruments and structured it across 18+ BCNF-normalized tables
    \item Built a Python-based ETL pipeline to automate data extraction, transform logs, and sync research data to a cloud warehouse
    \item \underline{Research:} Re-architected Python library for Raman Fitting Model ({\href{https://github.com/harshitjain17/Raman-Peak-Fitting-Model/}{try here}}) to implement spectra deconvolution, automate data preprocessing, and optimize export functions, reducing analysis time by 40\% and improving model accuracy
    \item \underline{Technologies Used:} Python, JavaScript, React.js, Node.js, MS SQL Server, RESTful APIs, Git, HTML/CSS, JIRA
\end{itemize}

\end{rSection} 


% ----------------------------------------------------------------------------------------
% 	PROJECT EXPERIENCE SECTION
% ----------------------------------------------------------------------------------------

\begin{rSection}{PROJECTS}

{\bf Dynamic Memory Allocator }{[C/C++] } \hfill Jan $2024$ - Feb $2024$
\begin{itemize}[itemsep = -4pt]
    \item Designed custom malloc, free, realloc; segregated free lists and footer optimization to improve memory management
    \item Achieved a utilization score of 69\% and benchmark throughput at 100\% across diverse computing environments
\end{itemize}

\end{rSection}
\end{document}